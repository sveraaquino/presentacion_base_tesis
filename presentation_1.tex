%%%%%%%%%%%%%%%%%%%%%%%%%%%%%%%%%%%%%%%%%
% Beamer Presentation
% LaTeX Template
% Version 1.0 (10/11/12)
%
% This template has been downloaded from:
% http://www.LaTeXTemplates.com
%
% License:
% CC BY-NC-SA 3.0 (http://creativecommons.org/licenses/by-nc-sa/3.0/)
%
%%%%%%%%%%%%%%%%%%%%%%%%%%%%%%%%%%%%%%%%%

%----------------------------------------------------------------------------------------
%	PACKAGES AND THEMES
%----------------------------------------------------------------------------------------

\documentclass[xcolor=table]{beamer}

\mode<presentation> {

% The Beamer class comes with a number of default slide themes
% which change the colors and layouts of slides. Below this is a list
% of all the themes, uncomment each in turn to see what they look like.

%\usetheme{default}
%\usetheme{AnnArbor}
%\usetheme{Antibes}
%\usetheme{Bergen}
%\usetheme{Berkeley}
%\usetheme{Berlin}
%\usetheme{Boadilla}
%\usetheme{CambridgeUS}
%\usetheme{Copenhagen}
%\usetheme{Darmstadt}
%\usetheme{Dresden}
%\usetheme{Frankfurt}
%\usetheme{Goettingen}
%\usetheme{Hannover}
%\usetheme{Ilmenau}
%\usetheme{JuanLesPins}
%\usetheme{Luebeck}
\usetheme{Madrid}
%\usetheme{Malmoe}
%\usetheme{Marburg}
%\usetheme{Montpellier}
%\usetheme{PaloAlto}
%\usetheme{Pittsburgh}
%\usetheme{Rochester}
%\usetheme{Singapore}
%\usetheme{Szeged}
%\usetheme{Warsaw}

% As well as themes, the Beamer class has a number of color themes
% for any slide theme. Uncomment each of these in turn to see how it
% changes the colors of your current slide theme.

%\usecolortheme{albatross}
%\usecolortheme{beaver}
%\usecolortheme{beetle}
%\usecolortheme{crane}
%\usecolortheme{dolphin}
%\usecolortheme{dove}
%\usecolortheme{fly}
%\usecolortheme{lily}
%\usecolortheme{orchid}
%\usecolortheme{rose}
%\usecolortheme{seagull}
%\usecolortheme{seahorse}
%\usecolortheme{whale}
%\usecolortheme{wolverine}

%\setbeamertemplate{footline} % To remove the footer line in all slides uncomment this line
%\setbeamertemplate{footline}[page number] % To replace the footer line in all slides with a simple slide count uncomment this line

%\setbeamertemplate{navigation symbols}{} % To remove the navigation symbols from the bottom of all slides uncomment this line
}

\usepackage{graphicx} % Allows including images
\usepackage[spanish]{babel}
\usepackage{booktabs} % Allows the use of \toprule, \midrule and \bottomrule in tables
 \usepackage{multirow}


%----------------------------------------------------------------------------------------
%	TITLE PAGE
%----------------------------------------------------------------------------------------

\title[Secuestro de Carbono]{Procesamiento de im\'agenes para determinar cambios del uso de la tierra y estimar secuestro de carbono en el Chaco Paraguayo} % The short title appears at the bottom of every slide, the full title is only on the title page

\author{Santiago Vera Aquino} % Your name
\institute[FP-UNA] % Your institution as it will appear on the bottom of every slide, may be shorthand to save space
{
Universidad Nacional de Asunci\'on \\ % Your institution for the title page
\medskip
\textit{sveraaquino@gmail.com} % Your email address
}
\date{\today} % Date, can be changed to a custom date

\begin{document}

\begin{frame}
\titlepage % Print the title page as the first slide
\end{frame}





\begin{frame}
\frametitle{Introducci\'on}
El cambio del paisaje de bosques tropicales maduros a paisajes agr\'icolas resulta en una emisi\'on neta de di\'oxido de carbono CO2 a la atm\'osfera. La deforestaci\'on es uno de los factores principales en el balance global de carbono, d\'onde la importancia de los ecosistemas forestales para mitigar las emisiones de gases como el di\'oxido de carbono, que es uno de los más importantes, radica en el potencial de fijaci\'on de estos en la biomasa de las especies arb\'oreas por medio de la fotos\'intesis; creando de este modo un reservorio importante que dependiendo del uso destinado podría almacenar el CO2 por un periodo de tiempo prolongado.\\~\\

Paraguay es un pa\'is que basa su econom\'ia en la agricultura y la ganader\'ia extensiva, actividades que han afectado al recurso forestal dando como resultado extensas \'areas deforestadas y degradadas.\\~\\

\end{frame}

\begin{frame}
	\frametitle{Introducci\'on}
	A pesar que existen leyes de protecci\'on para evitar la deforestaci\'on y valorar los bosques como la Ley de Deforestaci\'on Cero en la Regi\'on Oriental del Paraguay promulgada en el a\~{n}o 2004, y que ser\'a extendida hasta el 2018 y, la Ley de servicios ambientales 3001/06, entre otros instrumentos los mismos, necesitan apoyo para su monitoreo y aplicaci\'on efectiva.\\~\\
	
	Con el objetivo de implementar Pol\'iticas de mitigaci\'on del Cambio Cl\'im\'atico relativas a reducir las emisiones provenientes de la degradaci\'on y la deforestaci\'on (REDD+), los pa\'ises en desarrollo deben contar con estimaciones robustas s\'olidas en cuanto a las reservas de carbono forestal.
	
\end{frame}

%------------------------------------------------

\begin{frame}
	\frametitle{Formulaci\'on general del Proyecto}
	\begin{block}{Problema u Oporunidad}
		\begin{itemize}
			\item La informaci\'on referente al secuestro de carbono es muy escasa y discontinua en todo el pa\'is, m\'as aun en zonas del chaco, lo cual dificultad la detecci\'on y soluci\'on de las problem\'aticas que acarrea la p\'erdida de carbono.
			\item La generaci\'on de informaci\'on ambiental es muy costosa en el Paraguay, debido a no contar con una metodología pr\'actica  que agilice los procesos utilizando informaciones p\'ublicas y herramientas libres.
		\end{itemize}
	\end{block}
		\begin{block}{Soluci\'on propuesta por el proyecto de investigaci\'on}
			\begin{itemize}
				\item La soluci\'on propuesta por el proyecto es dise\~{n}ar e implementar una metodolog\'ia que ayude a estimar y comparar  el contenido de carbono en una regi\'on de manera pr\'actica y con un flujo continuo en series de tiempo.
			\end{itemize}
		\end{block}
	
\end{frame}

%------------------------------------------------
%------------------------------------------------

\begin{frame}
	\frametitle{Formulaci\'on general del Proyecto}
	\begin{block}{Hip\'otesis del proyecto}
		\begin{itemize}
			\item La idea consiste en aplicar operaciones de procesamiento de im\'agenes para realizar una comparaci\'on multitemporal de im\'agenes espectrales e \'indices derivados de ellas, previo a la clasificaci\'on de vegetaci\'on para la estimaci\'on de carbono.
		\end{itemize}
	\end{block}
	
\end{frame}
%------------------------------------------------

\begin{frame}
	\frametitle{Objetivos del proyecto}
	\begin{block}{Objetivos Generales}
		\begin{itemize}
			\item Desarrollar una metodolog\'ia de an\'alisis de im\'agenes espectrales multitemporales para la generaci\'on de indicadores  respecto a cambios de contenido de carbono en zonas del Chaco Paraguayo.
		\end{itemize}
	\end{block}
	\begin{block}{Objetivos Espec\'ifico}
		\begin{itemize}
			\item Estudiar el estado del arte en teledetecci\'on aplicada en el medio ambiente.
			\item Realizar detecciones de cambio automatizada dentro del \'area de estudio a trav\'es de la Teledetecci\'on y un SIG.
			\item Realizar	proyecciones de acuerdo a las tendencias observadas en los resultados.
			\item Aportar 	informaci\'on ambiental en la zona del chaco, para futuros estudios e investigaciones.
		\end{itemize}
	\end{block}
	
\end{frame}

%------------------------------------------------
%------------------------------------------------
\begin{frame}\frametitle{Antecedentes - Carbono y Biomasa}

La biomasa es aquella materia org\'anica de origen vegetal o animal, incluyendo los residuos y desechos org\'anicos, susceptible de ser aprovechada energ\'eticamente. Las plantas transforman la energ\'ia radiante del sol en energ\'ia qu\'imica a trav\'es de la fotos\'intesis, y parte de esta energ\'ia queda almacenada en forma de materia org\'anica.\\~\\
El conocer la cantidad de biomasa por \'arbol en cada una de las especies que crecen en el bosque, permite estimar la cantidad de carbono que contiene un grupo de \'arboles, un rodal o el bosque en su totalidad, ya que al multiplicar la cantidad de biomasa por un factor de conversi\'on, obtenido por muestreo, se logra determinar la cantidad de carbono.

\end{frame}
%------------------------------------------------
\begin{frame}\frametitle{Antecedentes - Teledetecci\'on}

		La teledetecci\'on es la ciencia y arte de obtener informaci\'on acerca de la superficie de la Tierra sin entrar en contacto con ella. Esto se realiza detectando y grabando la energ\'ia emitida o reflejada y procesando, analizando y aplicando esa informaci\'on.\\~\\
		Para producirse la teledetecci\'on, se requiere de la interacci\'on de tres componentes principales: flujo energ\'etico, objeto observado y un sensor.

\end{frame}


%------------------------------------------------
%------------------------------------------------
\begin{frame}\frametitle{Antecedentes - Teledetecci\'on}
\textbf{El espectro electromagn\'etico}\\~\\
Aunque los valores de longitudes de onda son continuos, se establece una serie de bandas en donde la radiación electromagnética manifiesta un comportamiento similar. El espectro electromagnético es la organización de estas bandas de longitudes de onda o frecuencia.
Las bandas m\'as empleadas en teledetecci\'on son las siguientes:
\begin{itemize}
\item Espectro visible (400 nm a 700 nm)
\item Infrarrojo próximo: (700 nm a 1300 nm)
\item Infrarrojo medio: (1,3 um a 8 um)
\item Infrarrojo lejano o térmico: (8 um a 14 um)
\item Microondas: (a partir de 1 um)
\end{itemize}

	
\end{frame}
%------------------------------------------------
%------------------------------------------------
\begin{frame}\frametitle{Antecedentes - Teledetecci\'on}
	\textbf{Firmas espectrales}\\~\\	

\begin{figure}
\centering
\includegraphics[width=0.7\linewidth]{imagenes/espectral}
\caption{Firma espectral de diferentes coberturas}
\label{fig:espectral}
\end{figure}

\end{frame}

%------------------------------------------------
\begin{frame}\frametitle{Antecedentes - Im\'agenes satelitales}
La imagen satelital consiste de un arreglo matricial bidimensional de elementos de imagen llamados p\'ixeles, ordenados en filas y columnas formando una malla las im\'agenes organizadas de esta manera son conocidas como im\'agenes r\'aster. Cada p\'ixel representa un \'area de superficie sobre la tierra. Un p\'ixel tiene un valor de intensidad y una ubicaci\'on en la imagen bidimensional.
	
\end{frame}
%------------------------------------------------
\begin{frame}\frametitle{Antecedentes - Resoluciones de un sensor}

	\begin{itemize}
		\item Resoluci\'on espacial
		\item Resoluci\'on espectral
		\item Resoluci\'on radiom\'etrica
		\item Resoluci\'on temporal
	\end{itemize}
	
	
\end{frame}

%------------------------------------------------
\begin{frame}\frametitle{Antecedentes - Teledetecci\'on y Biomasa }
	
La biomasa es un par\'ametro que no se puede obtener directamente desde im\'agenes de sat\'elite, de hecho no se puede medir directamente ni siquiera en campo. Con la teledetecci\'on se obtienen im\'agenes que permiten analizar la reflectividad de los lugares en diferentes regiones del espectro electrom\'agnético y esa informaci\'on se puede transformar en ecuaciones que estimen par\'ametros biol\'ogicos de esas regiones (y a su vez, esos par\'ametros biol\'ogicos transformarlos en biomasa).
	
	
\end{frame}
%------------------------------------------------

%------------------------------------------------
\begin{frame}\frametitle{Antecedentes - Teledetecci\'on y Biomasa }
	
	La biomasa es un par\'ametro que no se puede obtener directamente desde im\'agenes de sat\'elite, de hecho no se puede medir directamente ni siquiera en campo. Con la teledetecci\'on se obtienen im\'agenes que permiten analizar la reflectividad de los lugares en diferentes regiones del espectro electrom\'agnético y esa informaci\'on se puede transformar en ecuaciones que estimen par\'ametros biol\'ogicos de esas regiones (y a su vez, esos par\'ametros biol\'ogicos transformarlos en biomasa).
	
	
\end{frame}
%------------------------------------------------




\begin{frame}\frametitle{Metodolog\'ia }
\textbf{M\'etodos de comparaci\'on basados en operaciones o algoritmos de \'algebra de imagen}\\~\\	
Estos m\'etodos, se basan en operaciones sencillas por lo que se pueden implementar en un proceso no supervisado y se estructuran en tres etapas generales: pre-proceso, proceso de asignación (criterios de decisión) y post-proceso.	
	
\end{frame}
%------------------------------------------------
\begin{frame}\frametitle{Metodolog\'ia }
	\textbf{Datos disponibles}\\~\\	
	Se dispone de im\'agenes Landsat 5 y 7 con Path/Row 228/76 de fechas 1/26/1992 y 8/17/1999 para las validaciones de precisic\'ion y control de calidad, junto con una imagen de un estudio elaborado por University of Maryland Institute for Advanced Computer Studies respecto al cambio de vegetac\'ion.
	Para el calculo del Umbral de vegetaci\'on se utilizaron una imagen landsat 5 del a\~{n}o 1986 junto con MODIS Vegetation Continuous Fields elaborado por University of Maryland, Department of Geography and NASA, del a\~{n}o 2000.
	Por \'ultimo para la determinaci\'on del carbono una imagen landsat 7 y el mapa de carbono elaborado por California Institute of Technology perteneciente a la NASA, las dos im\'agenes del 2001.
	
\end{frame}
%------------------------------------------------

\begin{frame}\frametitle{Metodolog\'ia - Pre-Proceso}
	\textbf{Correci\'on Geom\'etricas}\\~\\	
Los ajustes geom\'etricos engloban toda serie de operaciones aplicadas sobre las	im\'agenes iniciales que permitan el co-registro espacial de estas; de manera que las celdas situadas en la misma posici\'on en cada imagen, se asocia a la misma área del terreno.
	
	
\end{frame}
%------------------------------------------------

\begin{frame}
	\frametitle{Metodolog\'ia - Pre-Proceso}
	\textbf{Correcci\'on radiom\'etrica}\\~\\	
	Se busca optimizar el proceso para mejorar la semejanza entre im\'agenes aplicando m\'etodos de normalizaci\'on a partir de los par\'ametros estad\'isticos de la imagen.\\~\\
	Una variable tipificada \textit{(Z)} se define para una distribuci\'on est\'andar del tipo \textit{N(0,1)} seg\'un la expresión:
	\begin{equation}
	Z=\dfrac{VD-\mu}{\sigma}
	\end{equation}
	Aplicando este concepto sobre cada una de las dos im\'agenes, se pueden comparar siendo ambas distribuciones estandarizadas:
	
\end{frame}
%------------------------------------------------
\begin{frame}
	\frametitle{Metodolog\'ia - Pre-Proceso}
	\begin{equation}
	\dfrac{VD_{1}-\mu_{1}}{\sigma_{1}}=\dfrac{VD_{2}-\mu_{2}}{\sigma_{2}}
	\end{equation}
	Para su aplicación pr\'actica, se puede transformar el valor digital (VD) de las celdas de la imagen 1, para que se asemeje al VD de las de la imagen 2, expresi\'on:
	\begin{equation}
	VD_{Norm}=\mu_{2}+\dfrac{\sigma_{2}}{\sigma_{1}}\cdot(VD_{1}-\mu_{1})
	\end{equation}
	Así se puede definir una relaci\'on lineal entre las dos distribuciones; aplicando una normalizaci\'on radiom\'etrica estad\'istica, los par\'ametros de la transformaci\'on \textit{m12} y \textit{n12} se definen seg\'un se indica en la expresi\'on:

	
\end{frame}
%------------------------------------------------
%------------------------------------------------
\begin{frame}
	\frametitle{Metodolog\'ia - Pre-Proceso}
	\begin{equation}
	n_{12}=\mu_{2}-\dfrac{\sigma_{2}}{\sigma_{1}}\cdot\mu_{1} ; 
	m_{12}=\dfrac{\sigma_{2}}{\sigma_{1}} \Rightarrow VD_{Norm}=m_{12}\cdot VD_{1}+n_{12}
	\end{equation}
	\begin{figure}
\centering
\includegraphics[width=0.8\linewidth]{imagenes/normalizacion}
\caption{Normalizaci\'on Radiom\'etrica}
\label{fig:normalizacion}
\end{figure}
	
\end{frame}

%------------------------------------------------
\begin{frame}
	\frametitle{Metodolog\'ia - Pre-Proceso}
	\textbf{Extracci\'on de \'Indices - NDVI}\\~\\	
	Los \'indices espectrales muestran un aspecto o caracter\'istica del terreno a partir de la informaci\'on radiom\'etrica contenida en las im\'agenes multiespectrales.\\~\\
	El \'indice diferencial de vegetaci\'on normalizado (NDVI) ha sido ampliamente reconocido como uno de los m\'as \'utiles para el estudio de caracter\'isticas de la biosfera terrestre y su din\'amica, a nivel global, regional y local. El NDVI es un \'indice espectral normalizado que toma valores en el intervalo $[1,-1]$ y se extrae de las bandas correspondientes al rojo $B_{R}$ e infrarrojo pr\'oximo $B_{IRc}$ seg\'un la siguiente expresi\'on:
	\begin{equation}
	NDVI=\dfrac{B_{IRc}-B_{R}}{B_{IRc}+B_{R}}
	\end{equation}
	
\end{frame}

%------------------------------------------------
\begin{frame}
	\frametitle{Metodolog\'ia - Proceso de Detecci\'on de cambio}
	\textbf{Comparaci\'on multitemporal}\\~\\	
	La diferencia de imagen por ser el método más simple, fácil de interpretar y directo; se suele aplicar combinada con la extracción de índices espectrales.
	\begin{equation}
	I_{Dif.}=VD_{final}-VD_{inicial}
	\end{equation}

\end{frame}
%------------------------------------------------
\begin{frame}
	\frametitle{Metodolog\'ia - Proceso de Detecci\'on de cambio}
	\textbf{Criterios de decisi\'on. Umbralizaci\'on.}\\~\\	
Se propone, como criterios de decisi\'on, aplicar m\'etodo de discriminaci\'on basado en los par\'ametros estad\'isticos del \'indice de cambio entre la secuencia temporal de im\'agenes:
	\begin{equation}
U=\mu\pm n\cdot\sigma
	\end{equation}
	Donde, el valor de umbral entre cambio/no cambio $(U)$ se estima en funci\'on de los par\'ametros estad\'isticos $(\mu,\sigma) $ y un coeficiente de tolerancia $ n $ asignado en funci\'on del tipo de datos disponibles (la fiabilidad del m\'etodo de captura realizado). Se clasifican los resultados en funci\'on de “n”; alta probabilidad de cambio $ (n\geq2) $  y zonas de media probabilidad de cambio $ (1<n<2) $.
	
\end{frame}
%------------------------------------------------
\begin{frame}
	\frametitle{Metodolog\'ia - Proceso de Detecci\'on de cambio}
	\textbf{Proceso Iterativo}\\~\\	
Al considerarse las dos im\'agenes como semejantes, los cambios producidos en el terreno afectan a la radiometr\'ia registrada en las im\'agenes, y por tanto, en los par\'ametros estad\'isticos que las definen.
Se propone realizar una normalizaci\'on radiom\'etrica iterativa, transformando la imagen a normalizar utilizando los par\'ametros estad\'isticos estimados a partir de las celdas clasificadas como no cambio. 
\end{frame}

%------------------------------------------------
\begin{frame}
	\frametitle{Metodolog\'ia - Proceso de Detecci\'on de cambio}
\begin{figure}
\centering
\includegraphics[width=0.8\linewidth, height=0.5\textheight]{imagenes/iteracion}
\caption{Diagrama de iteraci\'on}
\label{fig:iteracion}
\end{figure}


\end{frame}
%------------------------------------------------
\begin{frame}
	\frametitle{Metodolog\'ia - Post-Proceso}
	\textbf{Evaluaci\'on de los resultados y control de calidad}\\~\\	
Se realizaron el calculo del porcentaje de precisi\'on global y el coeficiente Kappa.
	
\end{frame}

%------------------------------------------------
\begin{frame}
	\frametitle{Metodolog\'ia - Post-Proceso}

% Please add the following required packages to your document preamble:
% If you use beamer only pass "xcolor=table" option, i.e. \documentclass[xcolor=table]{beamer}
\begin{table}[]
	\centering
	\begin{tabular}{|c|l|l|l|l|}
		\hline
		\rowcolor[HTML]{FFFFC7} 
		\multicolumn{1}{|l|}{\cellcolor[HTML]{FFFFC7}Coeficiente de tolerancia} & Zona   & GA        & Coef. Kappa & Puntos  \\ \hline
		& Urbana & 84.177819 & 0.474973    & 2835532 \\ \cline{2-5} 
		& Rural  & 93.874461 & 0.659546    & 3190560 \\ \cline{2-5} 
		\multirow{-3}{*}{N=1}                                                   & Humeda & 90.603407 & 0.299295    & 1865591 \\ \hline
		& Urbana & 83.47319  & 0.314602    & 2835532 \\ \cline{2-5} 
		& Rural  & 94.921675 & 0.673034    & 3190560 \\ \cline{2-5} 
		\multirow{-3}{*}{N=1.5}                                                 & Humeda & 95.278279 & 0.42258     & 1865591 \\ \hline
		& Urbana & 81.630537 & 0.093501    & 2835532 \\ \cline{2-5} 
		& Rural  & 94.334537 & 0.571205    & 3190560 \\ \cline{2-5} 
		\multirow{-3}{*}{N=2}                                                   & Humeda & 96.68802  & 0.425243    & 1865591 \\ \hline
	\end{tabular}
	\caption{Resultados de validaci\'on}
	\label{my-label}
\end{table}
	
\end{frame}

%------------------------------------------------
\begin{frame}
	\frametitle{Metodolog\'ia - Post-Proceso}
	\textbf{C\'alculo de Carbono}\\~\\	
	En base al mapa global de carbono correspondiente a Saatchi y a nuestro c\'alculo de NDVI de la misma fecha, se realizo un muestro observando una correlaci\'on. De esta forma es posible hallar una funci\'on lineal que convierta el NDVI a carbono.
	
\end{frame}

%------------------------------------------------
\begin{frame}
	\frametitle{Metodolog\'ia - Post-Proceso}
		\textbf{Muestreo con 240 puntos aleatorios del a\~{n}o 2001}\\~\\	
\begin{figure}
\centering
\includegraphics[width=0.7\linewidth, height=0.6\textheight]{imagenes/ndviCarb}
\caption{Regresi\'on Lineal con r2= 0.509125 (moderado)}
\label{fig:ndviCarb}
\end{figure}

	
\end{frame}




%------------------------------------------------

\begin{frame}
\Huge{\centerline{Gracias}}
\end{frame}

%----------------------------------------------------------------------------------------

\end{document} 